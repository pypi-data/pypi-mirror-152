\label{\detokenize{tabular:table-having-widths-option}}

\begin{savenotes}\sphinxattablestart
\centering
\phantomsection\label{\detokenize{tabular:namedtabular}}\label{\detokenize{tabular:mytabular}}\nobreak
\begin{tabular}[t]{|\X{30}{100}|\X{70}{100}|}
\hline
\sphinxstyletheadfamily 
\sphinxAtStartPar
header1
&\sphinxstyletheadfamily 
\sphinxAtStartPar
header2
\\
\hline
\sphinxAtStartPar
cell1\sphinxhyphen{}1
&
\sphinxAtStartPar
cell1\sphinxhyphen{}2
\\
\hline
\sphinxAtStartPar
cell2\sphinxhyphen{}1
&
\sphinxAtStartPar
cell2\sphinxhyphen{}2
\\
\hline
\sphinxAtStartPar
cell3\sphinxhyphen{}1
&
\sphinxAtStartPar
cell3\sphinxhyphen{}2
\\
\hline
\end{tabular}
\par
\sphinxattableend\end{savenotes}

\sphinxAtStartPar
See {\hyperref[\detokenize{tabular:mytabular}]{\sphinxcrossref{\DUrole{std,std-ref}{this}}}}, same as {\hyperref[\detokenize{tabular:namedtabular}]{\sphinxcrossref{namedtabular}}}.
