\documentclass{article}
\usepackage[utf8]{inputenc}

\title{Correlator Redis Keys}
\author{Jack Hickish, David DeBoer}
\date{June 2019, February 2020}

\begin{document}

\maketitle

\section{Introduction}
The HERA correlator implements a correlation pipeline using the HASHPIPE engine. The HASHPIPE library has built-in support for status monitoring using a redis database.
This document outlines the status keys the correlator posts to redis, most of which come from the original PAPER GPU correlator design.

\section{F-Engine keys}

The F-engine (~aka SNAPs) store information related to the mappings between SNAP, antenna, position as \texttt{corr:map<key>}, which may be retrieved by using \texttt{json.loads(<redis-session>.hget(`corr:map', <key>))}
The following keys under the \texttt{corr:map} hash are implemented

\begin{centering}
\begin{tabular}{c p{0.7\textwidth}}
\hline
Key Name & Description \\
\hline
\hline
ant\_to\_snap           & Mapping of antenna number to snap hostname \\
snap\_to\_ant           & Mapping of snap hostnames to antennas \\
all\_snap\_inputs     & Dictionary of all SNAP parts and ADC numbers \\
antenna\_numbers  & list of the associated antenna numbers for the antenna\_positions \\
antenna\_names & associated list of antenna names \\
antenna\_positions                   & uvutils.rotECEF\_from\_ECEF positions of all the antennas \\
antenna\_utm\_eastings & easting positions of antennas \\
antenna\_utm\_northings & northing positions of antennas \\
cofa                          & lat, lon, alt of the center-of-the array \\
cofa\_xyz                 & x, y, z of the center-of-the array \\
update\_time           & Unix time of update \\
update\_time\_str     & Human-readable time of update \\
snap\_host                & Mapping of SNAP part number to hostname \\
snap\_host\_update\_time & Unix time of snap hostname update \\
snap\_host\_update\_time\_str & Human-readable time of snap update\\
\end{tabular}
\end{centering}


\section{X-Engine keys}

X-engines store status hashes in redis using the keys: \texttt{hashpipe://<host>/<M>/status}, where `host` is the X-engine hostname (px\{1..16\}) and M is the pipeline instance (0..1) running on the machine.

Each status key may be queried using REDIS's \texttt{HGET} or \texttt{HGETALL} commands. Hash entries are as described in the following table. All entries are stored as strings.
\\
\subsection{General Hash Entries}

\begin{centering}
\begin{tabular}{c p{0.7\textwidth}}
\hline
Key Name & Description \\
\hline
\hline
GIT\_VER & The git hash of the running version of \texttt{paper\_gpu}. \\
INSTANCE & The instance ID of this hashpipe pipeline. \\
TIMEIDX  & The index (0, 1) of the time slices being processed by this pipeline. 0 = even, 1 = odd \\
XID      & The overall ID of this X-engine pipeline in the total system (0..31) \\
\end{tabular}
\end{centering}

\subsection{Net-thread Hash Entries}

\begin{centering}
\begin{tabular}{c p{0.7\textwidth}}
\hline
Key Name & Description \\
\hline
\hline
BINDHOST & The Ethernet interface the network thread is bound to. Eg. "eth3" \\
BINDPORT & The UDP port the network thread is bound to \\
MISSEDFE & The number of antennas not being received over the network \\
MISSEDPK & The number of packets missed in the last data block, after taking into account the number of missing antennas \\
NETBKOUT & The block ID of the last completed data block \\
NETDROPS & Number of dropped packets reported by the kernel in the last data block \\
NETDRPTL & A running total of the number of dropped packets reported by the kernel \\
NETGBPS  & The throughput (data + application header bits) achieved for the last processed data block, based on the time spend processing and receiving (but not waiting) \\
NETMCNT  & The MCNT of the current block being processed \\
NETPKTS  & ? Number of packets received in the last data block\\
NETPKTTL & ? Running total of the number of packets received \\
NETPRCMN & The minimum reported time (in ns) required to process a packet \\
NETPRCMX & The maximum reported time (in ns) required to process a packet \\
NETPRCNS & The average time (in ns) required to process a packet based on the last data block processed \\
NETRECMN & The minimum time (in ns) taken by the "get packet" call to return \\
NETRECMX & The maximum time (in ns) taken by the "get packet" call to return \\
NETRECNS & The average time (in ns) taken by the "get packet" call to return, based on the last data block processed \\
NETWATMN & The minimum time (in ns) spent waiting for a packet \\
NETWATMX & The maximum time (in ns) spent waiting for a packet \\
NETWATNS & The average time (in ns) spent waiting for a packet based on the last data block processed \\
NETSTAT  & A simple descriptor of the thread status: "running", "blocked out", "holding" \\
\end{tabular}
\end{centering}


\subsection{Fluff-thread Hash Entries}

\begin{centering}
\begin{tabular}{c p{0.7\textwidth}}
\hline
Key Name & Description \\
\hline
\hline
FLUFBKIN & The current hashpipe block ID being processed \\
FLUFGBPS & The throughput (in gigabits per second of input processed) of the last processed hashpipe block \\
FLUFMCNT & The MCNT of the current block being processed \\
FLUFMING & The minimum reported throughput (in gigabits per second of input processed) of all the blocks processed so far \\
FLUFSTAT & A simple descriptor of the thread status: "fluffing", "blocked in", "blocked out". TODO: check that these are actually the values \\
\end{tabular}
\end{centering}


\subsection{GPU-thread Hash Entries}

\begin{centering}
\begin{tabular}{c p{0.7\textwidth}}
\hline
Key Name & Description \\
\hline
\hline
GPUBLKIN & The current hashpipe block ID being processed \\
GPUDEV   & The ID of the GPU device being used by this thread \\
GPUDUMPS & A running total of the number of data dumps from the GPU \\
GPUGBPS  & The throughput (in gigabits per second of \textbf{8 bit} input processed) of the last processed GPU block. \\
GPUMCNT  & The MCNT of the current block being processed \\
GPUSTAT  & A simple descriptor of the thread status: "waiting", "processing", "blocked in", "blocked out" \\
INTCOUNT & The number of spectra being accumulated on the GPU \\
INTSYNC  & The MCNT to which integrations are synchronized \\
INTSTAT  & Used to turn on and off GPU processing. "on", "off". \\
\end{tabular}
\end{centering}


\subsection{Output-thread Hash Entries}

\begin{centering}
\begin{tabular}{c p{0.7\textwidth}}
\hline
Key Name & Description \\
\hline
\hline
\hline
OUTBLKIN & Current input block ID being processed \\
OUTBYTES & Number of bytes transmitted for last integration \\
OUTDUMPS & Running total of number of integrations transmitted \\
OUTMBPS  & Output data rate (in megabits per second) of last integration transmission \\
OUTSECS  & Time taken to send last integration \\
OUTSTAT  & Simple description of thread status "Processing", "blocked in"

\end{tabular}
\end{centering}

\section{Catcher Keys}

The correlator catcher(s) store status hashes in redis using the keys: \texttt{hashpipe://<host>/<M>/status}, where `host` is the catcher hostname (hera-sn\{1..2\}) and M is the pipeline instance (currently always 0) running on the machine.

Each status key may be queried using REDIS's \texttt{HGET} or \texttt{HGETALL} commands. Hash entries are as described in the following table. All entries are stored as strings.
\\

\subsection{General Hash Entries}

\begin{centering}
\begin{tabular}{c p{0.7\textwidth}}
\hline
Key Name & Description \\
\hline
\hline
\textbf{General} & \\
GIT\_VER & The git hash of the running version of \texttt{paper\_gpu}. \\
INSTANCE & The instance ID of this hashpipe pipeline. \\
SYNCTIME & The Unix time corresponding to MCNT = 0 \\
\end{tabular}
\end{centering}

\subsection{Net-thread Hash Entries}

\begin{centering}
\begin{tabular}{c p{0.7\textwidth}}
\hline
Key Name & Description \\
\hline
\hline
\textbf{Net thread} & \\
BINDHOST & The Ethernet interface the network thread is bound to. Eg. "eth3" \\
BINDPORT & The UDP port the network thread is bound to \\
MISSEDPK & The number of packets missed in the last data block, after taking into account the number of missing antennas \\
NETBKOUT & The block ID of the last completed data block \\
NETDROPS & Number of dropped packets reported by the kernel in the last data block \\
NETDRPTL & A running total of the number of dropped packets reported by the kernel \\
NETGBPS  & The throughput (data + application header bits) achieved for the last processed data block, based on the time spend processing and receiving (but not waiting) \\
NETMCNT  & The MCNT of the current block being processed \\
NETPKTS  & ? Number of packets received in the last data block\\
NETPKTTL & ? Running total of the number of packets received \\
NETPRCMN & The minimum reported time (in ns) required to process a packet \\
NETPRCMX & The maximum reported time (in ns) required to process a packet \\
NETPRCNS & The average time (in ns) required to process a packet based on the last data block processed \\
NETRECMN & The minimum time (in ns) taken by the "get packet" call to return \\
NETRECMX & The maximum time (in ns) taken by the "get packet" call to return \\
NETRECNS & The average time (in ns) taken by the "get packet" call to return, based on the last data block processed \\
NETWATMN & The minimum time (in ns) spent waiting for a packet \\
NETWATMX & The maximum time (in ns) spent waiting for a packet \\
NETWATNS & The average time (in ns) spent waiting for a packet based on the last data block processed \\
CNETSTAT  & A simple descriptor of the thread status: "running", "blocked out", "holding" \\
\end{tabular}
\end{centering}

\subsection{Disk-thread Hash Entries}
\begin{centering}
\begin{tabular}{c p{0.7\textwidth}}
\hline
Key Name & Description \\
\hline
\hline
\textbf{Disk Thread} & \\
DISKBKIN & The current hashpipe block ID being processed \\
DISKGBPS & The throughput (in gigabits per second of input processed) of the last processed hashpipe block \\
DISKMCNT & The MCNT of the current block being processed \\
DISKMING & The minimum reported throughput (in gigabits per second of input processed) of all the blocks processed so far \\
DISKSTAT & A simple descriptor of the thread status: "idle", "blocked in", "blocked out". TODO: check that these are actually the values \\
DISKTBNS & ? Average time (in ns) taken to transpose data, based on last block processed \\
DISKTMAX & ? Maximum time (in ns) taken to transpose a data block \\
DISKTMIN & ? Minimum time (in ns) taken to transpose a data block \\
DISKWBNS & ? Average time (in ns) taken to write a block to disk \\
DISKWMAX & ? Maximum time (in ns) taken to write a block to disk \\
DISKWMIN & ? Minimum time (in ns) taken to write a block to disk \\
DUMPMS   & ? Time (in ms) required to ?? TODO: Suspect some of the above write time definitions are wrong \\
HDF5TPLT & Path to the HDF5 header template file \\
INTTIME  & Number of spectra in each integration from the correlator \\
MSPERFILE & Target duration per output file (in ms) \\
NFILES   & Number of files to be written in this recording run \\
NDONEFIL & Number of files written so far in this recording run \\
TAG      & Ascii tag to be written into data files for this recording run \\
TRIGGER  & Control signal to trigger a new data collection run \\

\end{tabular}
\end{centering}

\end{document}
