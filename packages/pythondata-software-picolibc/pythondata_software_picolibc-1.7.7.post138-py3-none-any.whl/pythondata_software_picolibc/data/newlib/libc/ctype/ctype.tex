/*
Copyright (c) 1989 The Regents of the University of California.
All rights reserved.

Redistribution and use in source and binary forms, with or without
modification, are permitted provided that the following conditions
are met:
1. Redistributions of source code must retain the above copyright
notice, this list of conditions and the following disclaimer.
2. Redistributions in binary form must reproduce the above copyright
notice, this list of conditions and the following disclaimer in the
documentation and/or other materials provided with the distribution.
3. Neither the name of the University nor the names of its contributors
may be used to endorse or promote products derived from this software
without specific prior written permission.

THIS SOFTWARE IS PROVIDED BY THE REGENTS AND CONTRIBUTORS ``AS IS'' AND
ANY EXPRESS OR IMPLIED WARRANTIES, INCLUDING, BUT NOT LIMITED TO, THE
IMPLIED WARRANTIES OF MERCHANTABILITY AND FITNESS FOR A PARTICULAR PURPOSE
ARE DISCLAIMED.  IN NO EVENT SHALL THE REGENTS OR CONTRIBUTORS BE LIABLE
FOR ANY DIRECT, INDIRECT, INCIDENTAL, SPECIAL, EXEMPLARY, OR CONSEQUENTIAL
DAMAGES (INCLUDING, BUT NOT LIMITED TO, PROCUREMENT OF SUBSTITUTE GOODS
OR SERVICES; LOSS OF USE, DATA, OR PROFITS; OR BUSINESS INTERRUPTION)
HOWEVER CAUSED AND ON ANY THEORY OF LIABILITY, WHETHER IN CONTRACT, STRICT
LIABILITY, OR TORT (INCLUDING NEGLIGENCE OR OTHERWISE) ARISING IN ANY WAY
OUT OF THE USE OF THIS SOFTWARE, EVEN IF ADVISED OF THE POSSIBILITY OF
SUCH DAMAGE.
 */
@node Ctype
@chapter Character Type Macros and Functions (@file{ctype.h})
This chapter groups macros (which are also available as subroutines)
to classify characters into several categories (alphabetic,
numeric, control characters, whitespace, and so on), or to perform
simple character mappings.

The header file @file{ctype.h} defines the macros.
@menu
* isalnum::   Alphanumeric character predicate (isalnum, isalnum_l)
* isalpha::   Alphabetic character predicate (isalpha, isalpha_l)
* isascii::   ASCII character predicate (isascii, isascii_l)
* isblank::   Blank character predicate (isblank, isblank_l)
* iscntrl::   Control character predicate (iscntrl, iscntrl_l)
* isdigit::   Decimal digit predicate (isdigit, isdigit_l)
* islower::   Lowercase character predicate (islower, islower_l)
* isprint::   Printable character predicates (isprint, isprint_l. isgraph, isgraph_l)
* ispunct::   Punctuation character predicate (ispunct, ispunct_l)
* isspace::   Whitespace character predicate (isspace, isspace_l)
* isupper::   Uppercase character predicate (isupper, isupper_l)
* isxdigit::  Hexadecimal digit predicate (isxdigit, isxdigit_l)
* toascii::   Force integers to ASCII range (toascii, toascii_l)
* tolower::   Translate characters to lowercase (tolower, tolower_l)
* toupper::   Translate characters to uppercase (toupper, toupper_l)
* iswalnum::  Alphanumeric wide character predicate (iswalnum, iswalnum_l)
* iswalpha::  Alphabetic wide character predicate (iswalpha, iswalpha_l)
* iswblank::  Blank wide character predicate (iswblank, iswblank_l)
* iswcntrl::  Control wide character predicate (iswcntrl, iswcntrl_l)
* iswdigit::  Decimal digit wide character predicate (iswdigit, iswdigit_l)
* iswgraph::  Graphic wide character predicate (iswgraph, iswgraph_l)
* iswlower::  Lowercase wide character predicate (iswlower, iswlower_l)
* iswprint::  Printable wide character predicate (iswprint, iswprint_l)
* iswpunct::  Punctuation wide character predicate (iswpunct, iswpunct_l)
* iswspace::  Whitespace wide character predicate (iswspace, iswspace_l)
* iswupper::  Uppercase wide character predicate (iswupper, iswupper_l)
* iswxdigit:: Hexadecimal digit wide character predicate (iswxdigit, iswxdigit_l)
* iswctype::  Extensible wide-character test (iswctype, iswctype_l)
* wctype::    Compute wide-character test type (wctype, wctype_l)
* towlower::  Translate wide characters to lowercase (towlower, towlower_l)
* towupper::  Translate wide characters to uppercase (towupper, towupper_l)
* towctrans:: Extensible wide-character translation (towctrans, towctrans_l)
* wctrans::   Compute wide-character translation type (wctrans, wctrans_l)
@end menu

@page
@include ctype/isalnum.def

@page
@include ctype/isalpha.def

@page
@include ctype/isascii.def

@page
@include ctype/isblank.def

@page
@include ctype/iscntrl.def

@page
@include ctype/isdigit.def

@page
@include ctype/islower.def

@page
@include ctype/isprint.def

@page
@include ctype/ispunct.def

@page
@include ctype/isspace.def

@page
@include ctype/isupper.def

@page
@include ctype/isxdigit.def

@page
@include ctype/toascii.def

@page
@include ctype/tolower.def

@page
@include ctype/toupper.def

@page
@include ctype/iswalnum.def

@page
@include ctype/iswalpha.def

@page
@include ctype/iswcntrl.def

@page
@include ctype/iswblank.def

@page
@include ctype/iswdigit.def

@page
@include ctype/iswgraph.def

@page
@include ctype/iswlower.def

@page
@include ctype/iswprint.def

@page
@include ctype/iswpunct.def

@page
@include ctype/iswspace.def

@page
@include ctype/iswupper.def

@page
@include ctype/iswxdigit.def

@page
@include ctype/iswctype.def

@page
@include ctype/wctype.def

@page
@include ctype/towlower.def

@page
@include ctype/towupper.def

@page
@include ctype/towctrans.def

@page
@include ctype/wctrans.def

